% natbib guide (https://gking.harvard.edu/files/natnotes2.pdf)
% \citet     #textual citations, print the abbreviated author list
% \citet*    #textual citations, print the full author list
% \citep     #parenthetical citations, print the abbreviated author list
% \citep*    #parenthetical citations, print the full author list
% \citealt    #the same as \citet but without any parentheses.
% \citealp    #the same as \citep but without any parentheses. 
% \citeauthor{ale91}         #Alex et al.
% \citeauthor*{ale91}        #Alex, Mathew, and Ravi
% \citeyear{ale91}           #1991 
% \citeyearpar{ale91}        #(1991)

\documentclass[english, xcolor=dvipsnames, aspectratio=169]{beamer}

\usepackage{tabularx}
\usepackage{graphicx}
\usepackage{adjustbox}
\usepackage[english]{babel}
\usepackage[export]{adjustbox}
\usepackage{tikz}
% Justify text (package and function)
% \apptocmd{command}{code}{success}{failure}
\usepackage{ragged2e}
\apptocmd{\frame}{}{\justifying}{} 
% Justify text in \item
\newcommand{\itemj}{\item \justifying}

% If...else package
\usepackage{ifthen}

% Package to set transparent background image
\usepackage{tikz}

% Include image package
\usepackage{graphicx}
% Set default path for images
\graphicspath{ {./imgs/} }
% Set figure number when included
\setbeamertemplate{caption}[numbered]

% Bibliography packages
\usepackage[numbers,sorting=none]{natbib}
\bibliographystyle{unsrtnat}
% Define colors variables
\definecolor{darkblue}{rgb}{0.12, 0.139, 0.164} % primary color
\definecolor{grey}{rgb}{0.3686, 0.5255, 0.6235} % secondary color

% Set theme palette colors
\setbeamercolor{palette primary}{bg=darkblue,fg=white}
\setbeamercolor{palette secondary}{bg=darkblue,fg=white}
\setbeamercolor{palette tertiary}{bg=darkblue,fg=white}
\setbeamercolor{palette quaternary}{bg=darkblue,fg=white}
% strucure means itemize, enumerate, etc
\setbeamercolor{structure}{fg=darkblue} 

% Set bibliography colors
\setbeamercolor{bibliography item}{fg=darkblue}
\setbeamercolor{bibliography entry author}{fg=black}
\setbeamercolor{bibliography entry title}{fg=black}
\setbeamercolor{bibliography entry location}{fg=black}
\setbeamercolor{bibliography entry note}{fg=black}
% Replaces book icon in bibliography with enumeration
\setbeamertemplate{bibliography item}{[\theenumiv]}

% Table of contents style
% \setbeamertemplate{section in toc}[sections numbered]
% \setbeamertemplate{subsection in toc}[subsections numbered]
\setbeamertemplate{section in toc}[circle]
\setbeamertemplate{subsection in toc}[ball unnumbered]

% Header with navigation bar
\setbeamertemplate{headline}
{
    \leavevmode
    \hbox{
    \begin{beamercolorbox}[wd=\paperwidth,ht=2.5ex,dp=1.125ex]{palette quaternary}
    \insertsectionnavigationhorizontal{\paperwidth}{}{\hskip0pt plus1filll}
    \end{beamercolorbox} 
    }
}
% Footer with custom caption

\setbeamertemplate{footline}
{
    \leavevmode
    \hbox{
    \begin{beamercolorbox}[wd=.33\paperwidth,ht=2.6ex,dp=1ex,center]{palette quaternary}
    \usebeamerc{}\insertshortauthor\hspace*{1ex}
    \end{beamercolorbox}
    \begin{beamercolorbox}[wd=.33\paperwidth,ht=2.6ex,dp=1ex,center]{palette quaternary}
    \usebeamerfont{institute in head/foot}\insertshortinstitute
    \end{beamercolorbox}
    \begin{beamercolorbox}[wd=.33\paperwidth,ht=2.6ex,dp=1ex,center]{palette quaternary}
    \insertframenumber{} / \inserttotalframenumber
    \end{beamercolorbox}}
    \vskip0pt
    
}

% Global Background must be put in preamble
\usebackgroundtemplate
{
    \tikz\node{\includegraphics[height=\paperheight, width=\paperwidth]{background.pdf}};
}

% Subitem
\newcommand{\subitem}[1]{
    {\setlength\itemindent{15pt} \item[-] #1}
}
% One line command to print table of contents - two parameters for modes
\newcommand{\customToC}[2]
{
    \begin{frame}{Overview}
    \tableofcontents[#1,#2]
    \end{frame}
}

% Command to plot centered figure
% Parameters: #1=image name, #2=caption 
\newcommand{\includefigure}[2]
{
    \begin{figure}[h]
    \caption{#2}
    \centering
    \includegraphics[width=0.5\textwidth]{#1}
    \end{figure}
}

% Command to plot larger figure
% Parameters: #1=image name, #2=caption 
\newcommand{\includefigurelarger}[2]
{
    \begin{figure}[h]
    \caption{#2}
    \centering
    \includegraphics[width=0.85\textwidth]{#1}
    \end{figure}
}

% Command to plot medium-large figure
% Parameters: #1=image name, #2=caption 
\newcommand{\includefiguremediumlarger}[2]
{
    \begin{figure}[h]
    \caption{#2}
    \centering
    \includegraphics[width=0.6\textwidth]{#1}
    \end{figure}
}

% Command to plot smaller figure
% Parameters: #1=image name, #2=caption 
\newcommand{\includefiguresmaller}[2]
{
    \begin{figure}[h]
    \caption{#2}
    \centering
    \includegraphics[width=0.25\textwidth]{#1}
    \end{figure}
}

% Command to plot medium sized figure
% Parameters: #1=image name, #2=caption 
\newcommand{\includefiguremedium}[2]
{
    \begin{figure}[h]
    \caption{#2}
    \centering
    \includegraphics[width=0.40\textwidth]{#1}
    \end{figure}
}


% Command to set section name as variable (\renewcommand to update)
\newcommand{\sectiontitle}{}
% Command to set subsection name as variable (\renewcommand to update)
\newcommand{\subsectiontitle}{}


% Title page
\title{The eNanoMapper Ontology}
\subtitle{An application ontology to enable data integration for nanomaterial risk assessment}
\author{Javier Millán Acosta}
\institute{BiGCaT - Maastricht University}
% Date
\day=19\relax
\month=09\relax
\year=2022\relax

\begin{document}

\frame{\titlepage}

% Complete table of contents (ToC)
% \customToC{hideallsubsections}{}
\customToC{}{}

% Section name and highlighted ToC
\renewcommand{\sectiontitle}{Introduction}
\section{\sectiontitle}
\customToC{currentsection,hideothersubsections}{}

% Section name and highlighted ToC
\renewcommand{\subsectiontitle}{eNanoMapper}
\subsection{\subsectiontitle}
\begin{frame}{\subsectiontitle}
    \begin{itemize}
        \item \textbf{eNanoMapper} is a broader project which aims to address data and model interoperability challenges for engineered nanomaterial safety. \cite{jeliazkova_enanomapper_2015} 
        \item \textbf{The eNanoMapper ontology} is an application ontology and reuses parts of several ontologies to describe the full domain of nanomaterial safety assessment. \cite{hastings_enanomapper_2015}. It was continued by NanoCommons and NanoSolveIT.
    \end{itemize}

\includefigure{eNanoMapper_LOGO.png}{The eNanoMapper logo}
\end{frame}

% 
\renewcommand{\subsectiontitle}{Engineered nanomaterials and safety}
\subsection{\subsectiontitle}
\begin{frame}{\subsectiontitle}
 			\begin{itemize}
\item Engineered nanomaterials (eNMs) are broadly defined as compounds that exist on a scale of ~1–100 nm. in at least one of their dimensions. \cite{hastings_enanomapper_2015}
\item Their safety assessment must cover the identification of:
\subitem{Physicochemical properties}
\subitem{Biological properties and activity} \cite{jeliazkova_enanomapper_2015}
\subitem{Diffusion behaviour into the natural environment} \cite{hastings_enanomapper_2015}
    		\end{itemize}

\end{frame}

% 
\renewcommand{\subsectiontitle}{Ontologies}
\subsection{\subsectiontitle}
\begin{frame}{\subsectiontitle}
 			\begin{itemize}
\item A formal representation of a set of concepts and the rules constraining how they become structured within a knowledge domain.  \cite{krotzsch_description_2013}  
\item They can be used to provide metadata (a computer-readable set of information that describes other data), or to reason over a domain.
\item It consists of three syntactical categories: \textbf{Entities}, \textbf{Expressions} and \textbf{Axioms}, which can be given annotations for further description. \cite{noauthor_owl_nodate}
    		\end{itemize}

\end{frame}




% 


\begin{frame}{\subsectiontitle}
\begin{itemize}
\item All entities (classes, object properties, named individuals...) are uniquely identified by a sequence of characters called IRI (International Resource Identifier, an extension of URI).

\end{itemize}

 \includefigurelarger{class.pdf}{An \textt{owl:Class} in an ontology text file (the ontology document). IRIs in blue.}
\end{frame}

%

\begin{frame}{\subsectiontitle}
    \begin{figure}
        \begin{minipage}[b]{0.55\linewidth}
 			\begin{itemize}
\item Most ontologies use the W3C standard language for ontologies, Web Ontology Language \textbf{OWL}.
\item OWL ontologies are mainly stored in \texttt{.owl} files, which are a sort of \textbf{RDF} document.
\item \textbf{RDF} (Resource Description Framework) is a standard for data exchange. It defines \textbf{triples} of \textbf{(subject, predicate, object)}. \cite{noauthor_rdf_nodate}
\item These triples form labeled graphs where the edge (predicate) represents the link between two resources (subject and object)
    		\end{itemize}
        \end{minipage}
    \hfill
     \begin{minipage}[b]{0.4\linewidth}
            \centering
            \includefigurelarger{pizzaontology.png}{The pizza ontology, visualized as a graph \cite{drummond_pizza_nodate}}
        \end{minipage}
    \end{figure}
\end{frame}


% 
\renewcommand{\subsectiontitle}{Ontologies - example of a class}
\subsection{\subsectiontitle}
\begin{frame}{\subsectiontitle}
\begin{figure}
    
    \centering
	\includefiguremediumlarger{class-hierarchy.png}{A view of the eNanoMapper ontology class hierarchy, showing the class "electric potential"}
    \end{figure}
\end{frame}

% 
\begin{frame}{\subsectiontitle}

     
\begin{figure}
    
    \centering
	\includefigurelarger{rdf-graph-triple.pdf}{A graph with two nodes (Subject and Object) and a triple connecting them (Predicate): the class with class description "electric potential" is a subclass of disposition.}
    \end{figure}
\end{frame}


% 
\begin{frame}{\subsectiontitle}

     
\begin{figure}
    \centering
	\includefigurelarger{rdf-owl.pdf}{The triple in the previous figure as expressed in the \texttt{.owl} document file of the ontology it is contained in.}
    \end{figure}
\end{frame}

% 

\begin{frame}{\subsectiontitle}
    \begin{itemize}
        \item \textbf{Foundation/Upper/Top-level ontologies}
        \item \textbf{Application and domain ontologies} 
    \end{itemize}
\end{frame}

% Section name and highlighted ToC
\renewcommand{\sectiontitle}{eNanoMapper Ontology}
\section{\sectiontitle}
\customToC{currentsection,hideothersubsections}{}



% 
\renewcommand{\subsectiontitle}{Design of the eNanoMapper ontology}
\subsection{\subsectiontitle}

\begin{frame}{\subsectiontitle}
\includefigure{enm-structure.png}{An overview of the upper levels, imports and manually annotated content going into the eNanoMapper ontology \cite{hastings_enanomapper_2015}}

\end{frame}

% 
\begin{frame}{\subsectiontitle}
\includefigure{enm-imports.png}{Distribution of the number of classes imported from each ontology into eNM \cite{laurent_winckers_2020_4032809}}

\end{frame}

% 

\begin{frame}{\subsectiontitle}
\centering
    The current approach uses in-house software (\textit{Slimmer}) to slim imports, and ROBOT to extract annotations.
    \begin{tikzpicture}[node distance={37mm}, thick, main/.style = {draw, square}] 
    \node[main] (1) {eNanoMapper Ontology}; 
    \node[main] (2) [above right of=1] {Internal (manually added) terms}; 
    \node[main] (3) [below right of=1] {External terms}; 
    \node[main] (4) [right of=1] {Annotations};
    \node[main] (5) [right of=4] {External ontologies}; 
    \draw[->] (2) -- node[midway, sloped, above left, pos=0.1]{\textit{owl:Imports}} (1); 
    \draw[->] (3) -- node[midway, sloped, above left, pos=0.1]{\textit{owl:Imports}} (1); 
    \draw[->] (4) -- (1); 
    \draw[->] (5) -- node[midway, sloped, above left, pos=0.4]{\textit{Slimmer}} (3); 
    \draw[->] (5) -- node[midway, above left, pos=0]{\textit{ROBOT}} (4); 
    \draw[->] (2) -- (4);
\end{tikzpicture} 
\end{frame}

% 

\begin{frame}{\subsectiontitle}
\begin{itemize}
    \item \textbf{SLIMMER} Uses the OWLAPI to create slims of the ontologies to be imported, and determine the class hierarchy of the imports. \cite{hastings_enanomapper_2015}
    \item \textbf{ROBOT} is a wrapper for the OWL API used by developers in the Open Biomedical Ontologies (among others) for development and quality control. \cite{jackson_robot_2019}
\end{itemize}
\end{frame}


% 

\begin{frame}{\subsectiontitle}
\centering
\includefiguremediumlarger{jenkins.pdf}{The current Jenkins setup for eNanoMapper ontology CI/CD}
\end{frame}



%
\renewcommand{\sectiontitle}{Migrating the eNM ontology CI/CD}
\section{\sectiontitle}
\customToC{currentsection,hideothersubsections}{}




% 
\renewcommand{\subsectiontitle}{Migrating the ontology development}
\subsection{\subsectiontitle}
\begin{frame}{\subsectiontitle}
\begin{itemize}
    \item Current setup: Jenkins for CI in own server (to be dropped soon)
    \item Alternatives:
        \subitem{Literal migration from Jenkins to GitHub actions}
        \subitem{Migration to the Ontology Development Kit-ROBOT - rethinking the whole process}
\end{itemize}
\end{frame}



% 
\renewcommand{\subsectiontitle}{A. Migrating current setup to GitHub actions}
\subsection{\subsectiontitle}
\begin{frame}{\subsectiontitle}
\centering
\includefiguremediumlarger{ghactionsrun.pdf}{A test run for the GitHub actions-based setup for eNanoMapper ontology CI}
\end{frame}

% 
\begin{frame}{\subsectiontitle}
\begin{itemize}
    \item  A toolkit developed for the development of biomedical ontologies following the software development approach (a method preceded by the eNanoMapper/jenkins/Slimmer setup)
    \item It is delivered in a docker image and uses ROBOT (a wrapper for the OWL API) and a series of scripts to standardize and automate steps like preparing ontology releases, continuous quality control checking, and dependency management. \cite{odk_matentzoglu}
\end{itemize}

    \includefiguremedium{odk-logo_black-banner.png}{The ODK logo}
\end{frame}

\begin{frame}{\subsectiontitle}
The challenge: replicating the same class hierarchies after substituting Slimmer with ROBOT.
\end{frame}

% 
\renewcommand{\subsectiontitle}{B. Ontology Development Kit}
\subsection{\subsectiontitle}
\begin{frame}{\subsectiontitle}
\centering
    The ODK approach relies on ROBOT to handle ontologies and serialization of several steps through standardized Makefiles. \cite{matentzoglu_ontology_2021}
    \begin{tikzpicture}[node distance={37mm}, thick, main/.style = {draw, square}] 
    \node[main] (1) {eNanoMapper Ontology}; 
    \node[main] (2) [above right of=1] {Internal (manually added) terms}; 
    \node[main] (3) [below right of=1] {External terms}; 
    \node[main] (4) [right of=1] {Annotations};
    \node[main] (5) [right of=4] {External ontologies}; 
    \draw[->] (2) -- node[midway, sloped, above left, pos=0.1]{\textit{owl:Imports}} (1); 
    \draw[->] (3) -- node[midway, sloped, above left, pos=0.1]{\textit{owl:Imports}} (1); 
    \draw[->] (4) -- (1); 
    \draw[->] (5) -- node[midway, sloped, above left, pos=0.4]{\textit{ROBOT}} (3); 
    \draw[->] (5) -- node[midway, above left, pos=0]{\textit{ROBOT}} (4); 
    \draw[->] (2) -- (4);
\end{tikzpicture} 
\end{frame}


% 
\renewcommand{\subsectiontitle}{Comparison}
\subsection{\subsectiontitle}
\begin{frame}{\subsectiontitle}
\begin{table}[ht]
\caption{Comparison of the two alternatives for eNanoMapper CI/CD}
\label{tab:Comparing-approaches}
  \begin{adjustbox}{width=\textwidth}


\begin{tabular}{lll}
                                                                                  & \textbf{GitHub actions}                                            & \textbf{Ontology Development Kit}             \\
\textbf{Uses tools}                                                               & GH actions, slimmer, maybe ROBOT for annotations, in-house scripts & ROBOT and ODK scripts                         \\
\textbf{Keeps current class hierarchy}                                            & Yes                                                                & Perhaps, but will be a challenge              \\
\textbf{Sticks to OBO recommendations} & Yes, after minimal adaptations if Slimmer is kept                                    & Yes                                           \\
\textbf{Hosted on}                                                                & GitHub                                                             & GitHub                                        \\
\textbf{Time to develop}                                                          & Ready at any point                                                 & Not optimal for migrating existing ontologies \cite{matentzoglu_ontology_2021}
\end{tabular}
  \end{adjustbox}
\end{table}
\end{frame}

% Import bibliography from file sample.bib
\begin{frame}[t, allowframebreaks]
\frametitle{References}

\bibliography{references}
\end{frame}

\end{document}